\documentclass[12pt,a4paper,oneside]{article}

\usepackage[utf8]{inputenc}
\usepackage[portuguese]{babel}
\usepackage[T1]{fontenc}
\usepackage{amsmath}
\usepackage{amsfonts}
\usepackage{amssymb}
\usepackage{graphicx}
\usepackage{xcolor}
% Definindo novas cores
\definecolor{verde}{rgb}{0.25,0.5,0.35}

\author{\\Universidade Federal de Goiás (UFG) - Câmpus Jataí\\Bacharelado em Ciência da Computação \\Teoria da Computação \\Esdras Lins Bispo Jr.}

\date{14 de maio de 2014}

\title{\sc \huge Segundo Teste}

\begin{document}

\maketitle

{\bf ORIENTAÇÕES PARA A RESOLUÇÃO}

\begin{itemize}
	\item A avaliação é individual, sem consulta;
	\item A pontuação máxima desta avaliação é 10,0 (dez) pontos, sendo uma das 06 (seis) componentes que formarão a média final da disciplina: quatro testes, uma prova e exercícios;
	\item A média final ($MF$) será calculada assim como se segue
	\begin{eqnarray}
		MF & = & MIN(10, S) \nonumber \\
		S & = & (\sum_{i=1}^{4} 0,2.T_i ) + 0,2.P  + 0,1.E \nonumber
	\end{eqnarray}
	em que 
	\begin{itemize}
		\item $S$ é o somatório da pontuação de todas as avaliações,
		\item $T_i$ é a pontuação obtida no teste $i$,
		\item $P$ é a pontuação obtida na prova, e
		\item $E$ é a pontuação total dos exercícios.
	\end{itemize}
	\item O conteúdo exigido desta avaliação compreende o seguinte ponto apresentado no Plano de Ensino da disciplina: (1) Teoria da Computação e (2) Modelos de Computação.
\end{itemize}

\begin{center}
	\fbox{\large Nome: \hspace{10cm}}
	\fbox{\large Assinatura: \hspace{9cm}}
\end{center}

\newpage

\begin{enumerate}
	
	\section*{Segundo Teste}
	
	\item (5,0 pt) Apresentamos logo abaixo a definição formal de uma máquina de Turing:
	
	\begin{center}
	\line(1,0){250}
	\end{center}	
	
	Uma {\bf máquina de Turing} é uma 7-upla $(Q, \Sigma, \Gamma, \delta, q_0, q_{aceita}, q_{rejeita})$, de forma que $Q, \Sigma, \Gamma$ são todos conjuntos finitos e
	
		\begin{itemize}
			\item $Q$ é o conjunto de estados,
			\item $\Sigma$ é o alfabeto de entrada sem o {\bf símbolo branco} $\sqcup$,
			\item $\Gamma$ é o alfabeto da fita, em que $\sqcup \in \Gamma$ e $\Sigma \subset \Gamma$,
			\item $\delta : Q \times \Gamma \rightarrow Q \times \Gamma \times \{E, D\}$ é a função de transição,
			\item $q_0 \in Q$ é o estado inicial,
			\item $q_{aceita} \in Q$ é o estado de aceitação, e
			\item $q_{rejeita} \in Q$ é o estado de rejeição, em que $q_{rejeita} \not= q_{aceita}$.
		\end{itemize}
		
	\begin{center}
	\line(1,0){250}
	\end{center}		
		
	 Responda às seguintes perguntas, justificando a sua resposta.
	\begin{enumerate}
		\item (1,0 pt) Uma máquina de Turing pode alguma vez escrever o símbolo branco $\sqcup$ em sua fita? \\
			{\color{verde}
				R - Sim, ela pode. Pois $\sqcup \in \Gamma$ (em que $\Gamma$ é o alfabeto da fita).
			}
		\item (1,5 pt) O alfabeto da fita $\Gamma$ pode ser o mesmo que o alfabeto de entrada $\Sigma$? \\
			{\color{verde}
				R - Não, não pode. Pois $\sqcup \in \Gamma$, mas $\sqcup \not\in \Sigma$. Logo, $\Gamma \not= \Sigma$.
			}
		\item (1,0 pt) A cabeça de uma máquina de Turing pode alguma vez estar na mesma localização em dois passos sucessivos?\\
			{\color{verde}
				R - Não, não pode. De acordo com a função $\delta$, uma dupla de entrada é mapeada em uma tripla (estado, símbolo da fita, movimento da cabeça). Logo, o movimento da cabeça é obrigatório, forçando o movimento ou para a esquerda ou para a direita, não permitindo que a cabeça permaneça na mesma célula.
			}
		\item (1,5 pt) Uma máquina de Turing pode conter apenas um único estado?\\
			{\color{verde}
				R - Não, não pode. Como o $q_{rejeita} \not= q_{aceita}$, então existe pelo menos dois estados distintos.
			}
	\end{enumerate}
	
	\item (5,0 pt) Mostre que se $A$ e $B$ são duas linguagens decidíveis, então a linguagem $\overline{A} \cup \overline{B}$ também é decidível.\\
			\\{\color{verde}	
				{\bf Prova:} Seja $M_A$ e $M_B$ duas máquinas de Turing que decidem as linguagens $A$ e $B$, respectivamente (pois se uma linguagem é decidível, então uma máquina de Turing a decide). Iremos construir a máquina de Turing $M_{aux}$, a partir de $M_A$ e $M_B$, que decide $\overline{A} \cup \overline{B}$. A descrição de $M_{aux}$ é dada a seguir:
			
			$M_{aux}$ = ``Sobre a entrada $\omega$, faça:
			\begin{enumerate}
				\item Rode $M_A$ sobre $\omega$. Se $M_A$ rejeita, {\it aceite}.
				\item Rode $M_B$ sobre $\omega$. Se $M_B$ rejeita, {\it aceite}.
				\item Se $M_A$ e $M_B$ aceitam, {\it rejeite}''.
			\end{enumerate}
			
			Como é possível construir $M_{aux}$, então $\overline{A} \cup \overline{B}$ é decidível. $\blacksquare$
			}

\end{enumerate}

\end{document}