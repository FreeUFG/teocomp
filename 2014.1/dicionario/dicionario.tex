\documentclass[12pt,a4paper,oneside]{article}

\usepackage[utf8]{inputenc}
\usepackage[portuguese]{babel}
\usepackage[T1]{fontenc}
\usepackage{amsmath}
\usepackage{amsfonts}
\usepackage{amssymb}

\usepackage{xcolor}
% Definindo novas cores
\definecolor{verde}{rgb}{0.25,0.5,0.35}
\definecolor{jpurple}{rgb}{0.5,0,0.35}

\author{\\Universidade Federal de Goiás (UFG) - Câmpus Jataí\\Bacharelado em Ciência da Computação \\Teoria da Computação \\Prof. Esdras Lins Bispo Jr.}
\date{}

\title{\sc \huge Dicionário de Demonstrações}

\begin{document}

\maketitle

\section{Livro de Referência}
	\begin{itemize}
		\item SIPSER, M. {\bf Introdução à Teoria da Computação}, 2a Edição, Editora Thomson Learning, 2011. \color{blue}{\bf Código Bib.: [004 SIP/int]}.
	\end{itemize}
	
	\subsection{Abreviaturas}
		\begin{itemize}
			\item[] {\bf AFD}: Autômato Finito Determinístico;
			\item[] {\bf AFN}: Autômato Finito Não-Determinístico.
		\end{itemize}
	
\section{Definições e Demonstrações}

\begin{center}
\line(1,0){250}
\end{center}

\begin{flushleft}
	\begin{tabular}{ll}
		{\bf Definição 1.16:} & Uma linguagem é chamada de uma linguagem regular \\
							& se algum autômato finito a reconhece.\\
	\end{tabular}
\end{flushleft}

\begin{center}
\line(1,0){250}
\end{center}

\begin{flushleft}
	\begin{tabular}{ll}
		{\bf Teorema 1.25:} & A classe de linguagens regulares é fechada sob a \\
							& operação da união.
	\end{tabular}
\end{flushleft}

{\bf Prova:} Sejam $A$ e $B$ duas linguagens regulares. A classe de linguagens regulares é fechada sob a operação de união se $A \cup B$ é regular. $A \cup B$ é regular se for possível construir um AFD $M$ que a reconheça (Definição 1.16).

Seja $M_A = ( Q_A, \Sigma, \delta_A, q_A, F_A )$ e $M_B = ( Q_B, \Sigma, \delta_B, q_B, F_B )$ os dois AFDs que reconhecem as linguagens $A$ e $B$, respectivamente (Definição 1.16). Iremos construir o AFD $M = ( Q, \Sigma, \delta, q_0, F )$ a partir de $M_A$ e $M_B$. $M$ pode ser construído como se segue:

\begin{enumerate}
	\item $Q = Q_A \times Q_B$;
	\item $\Sigma$ (o mesmo alfabeto para ambas as máquinas)\footnote{Embora seja admitido aqui que tanto $M_1$ quanto $M_2$ tem alfabetos iguais, o teorema ainda permanece verdadeiro caso contrário.};
	\item $\delta$, a função de transição, é definida da seguinte maneira. Para cada estado $(r_1,r_2) \in Q$ e cada $a \in \Sigma$, faça
		\begin{center}
			$\delta( (r_1, r_2), a ) = (\delta(r_1,a), \delta(r_2,a)) $;
		\end{center}
	\item $q_0$ é o par $(q_A, q_B)$;
	\item $F = \{ (r_1, r_2) \mbox{ | } r_1 \in F_A \vee r_2 \in F_B \}$.
\end{enumerate}

Como é possível construir $M$, então $A \cup B$ é regular. Logo, a classe de linguagens regulares é fechada sob a operação de união. $\blacksquare$

\begin{center}
\line(1,0){250}
\end{center}

\begin{flushleft}
	\begin{tabular}{cl}
		{\bf Teorema 1.25:} & A classe de linguagens regulares é fechada sob a \\
		{\bf [Modificado]}	& operação de intersecção.
	\end{tabular}
\end{flushleft}

{\bf Prova:} Sejam $A$ e $B$ duas linguagens regulares. A classe de linguagens regulares é fechada sob a operação de intersecção se $A \cap B$ é regular. $A \cap B$ é regular se for possível construir um AFD $M$ que a reconheça (Definição 1.16).

Seja $M_A = ( Q_A, \Sigma, \delta_A, q_A, F_A )$ e $M_B = ( Q_B, \Sigma, \delta_B, q_B, F_B )$ os dois AFDs que reconhecem as linguagens $A$ e $B$, respectivamente (Definição 1.16). Iremos construir o AFD $M = ( Q, \Sigma, \delta, q_0, F )$ a partir de $M_A$ e $M_B$. $M$ pode ser construído como se segue:

\begin{enumerate}
	\item $Q = Q_A \times Q_B$;
	\item $\Sigma$ (o mesmo alfabeto para ambas as máquinas);
	\item $\delta$, a função de transição, é definida da seguinte maneira. Para cada estado $(r_1,r_2) \in Q$ e cada $a \in \Sigma$, faça
		\begin{center}
			$\delta( (r_1, r_2), a ) = (\delta(r_1,a), \delta(r_2,a)) $;
		\end{center}
	\item $q_0$ é o par $(q_A, q_B)$;
	\item $F = \{ (r_1, r_2) \mbox{ | } r_1 \in F_A \wedge r_2 \in F_B \}$.
\end{enumerate}

Como é possível construir $M$, então $A \cap B$ é regular. Logo, a classe de linguagens regulares é fechada sob a operação de intersecção. $\blacksquare$

\begin{center}
\line(1,0){250}
\end{center}

\begin{flushleft}
	\begin{tabular}{ll}
		{\bf Teorema 1.39:} & Todo autômato finito não-determinístico tem um \\
							& autômato finito determinístico equivalente.
	\end{tabular}
\end{flushleft}

{\bf Prova:} Seja $N = (Q, \Sigma, \delta, q_0, F)$ o AFN que reconhece alguma linguagem $A$. Construiremos o AFD $M = (Q', \Sigma, \delta', q_0', F' )$ que reconhece $A$. Consideraremos, provisoriamente, o caso em que $N$ não tem setas $\epsilon$. Retornaremos a este caso mais adiante. $M$ pode ser construído como se segue:

	\begin{enumerate}
		\item $Q' = \mathcal{P}(A)$;
		\item $\Sigma$ (o mesmo alfabeto de $N$);
		\item $\delta'$, a função de transição, é definida da seguinte maneira. Para $R \in Q'$ e $a \in \Sigma$, faça 
			\begin{center}
				$\delta'(R,a) = \bigcup\limits_{r \in R} \delta (r,a)$
			\end{center}
		\item $q_0' = \{ q_0 \}$
		\item $F' = \{ R \in Q' \mbox{ | } R$ contém um estado de aceitação de $N \}$.
	\end{enumerate}

Consideraremos agora o caso envolvendo as setas $\epsilon$. Para qualquer estado $R$ de $M$, definimos $E(R)$ como a coleção de estados que podem ser atingidos a partir de $R$ indo somente ao longo de suas setas $\epsilon$, incluindo os próprios membros de $R$. Formalmente, para $R \subseteq Q$ seja
	\begin{center}
		$E(R) = \{q \mbox{ | } q$ pode ser atingido a partir de $R$ viajando-se ao longo de 0 ou mais setas $\epsilon \}$	
	\end{center}	
	Basta substituir na função de transição os termos $\delta(r,a)$ por $E(\delta(r,a))$. Temos
			\begin{center}
				$\delta'(R,a) = \bigcup\limits_{r \in R} E(\delta (r,a))$
			\end{center}
Também necessitamos modificar o estado inicial de $\{ q_0 \}$ para $E(\{ q_0 \})$. Assim, conseguimos construir $M$ que simula $N$. $\blacksquare$

\end{document}